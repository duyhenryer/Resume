% Awesome CV LaTeX Template
%
% This template has been downloaded from:
% https://github.com/huajh/huajh-awesome-latex-cv
%
% Author:
% Junhao Hua


%Section: Work Experience at the top
\sectionTitle{Projects \& Experiences}{\faCode}
 
\begin{experiences}
	
		
 \experience
    {May 2015}   {Computer vision and image processing}{ZJU }{C/Matlab/Python}
    {Oct 2013} {
                      \begin{itemize}
                        \item \emph{Object Recognition} based on SIFT feature implemented by Matlab mixed with C.
                        \item \emph{Recommender Systems} based on latent factor models and matrix factorization.                  
                        \item Implementation of \emph{{Image Seamless Editing}} by solving Poisson equations.
                        \item \emph{Image Denoising} based on non-linear anistropic diffusion techniques.
                        \item \faGithub: 
                        \link{https://github.com/huajh/sift}{sift}, 
                        \link{https://github.com/huajh/mf_re_sys}{MFResys},
                        \link{https://github.com/huajh/Poisson_image_editing}{PoissonImageEditing},
                        \link{https://https://github.com/huajh/Image_denoising}{ImageDenoising}.                                                                                       
                      \end{itemize}
                    }
                    {Object Recognition, Image Processing, Recommender Systems, Python}
  \emptySeparator
  \experience
    {Apr 2014} {Action/Behavior Recognition in Videos}{ZJU}{ Matlab}
    {Feb 2014}    {
                      \begin{itemize}
                        \item Extract the spatio-temporal features and obtain "Bag of words" represetation by clustering (k-means) the extracted features; 
                        \item Infer the posterior by pLSA/LDA (unsupervised Learning) or by simple classfications (KNN, SVM);                    
                        \item Propose a simple method called 'voting' to achieve multiple actions recogintion task.
                        \item \faGithub: \link{https://github.com/huajh/action_recognition} {github.com/huajh/action\_recognition}                                                                                          
                      \end{itemize}
                    }
                    {Action Recoginition, Machine Learning, Clustering, LDA, "Bag of Words" Representation}

  \emptySeparator
  \experience
  {May 2013} {Brain MR image segmentation}{ZJUT}{Bachelor Thesis}
  {Dec 2012 }    {
				  	\begin{itemize}
				  		\item Apply the GMM, student-t mixture model, and Dirichlet process based infinite mixture modelto the brain MR image clustering problem; 
				  		\item Derive the detail variational Bayesian inference process.
				  		\item Improve these three algorithms by using laplacian graph (manifold learning); 				  	              
				  		\item \faGithub: \link{https://github.com/huajh/variational_bayesian_clusterings} {github.com/huajh/variational\_bayesian\_clusterings}                                                                                    
				  	\end{itemize}
				  }
				  {Mixture Model, Clustering, Dirichlet Process, Variational Bayes, Manifold Learnig}
				  
  \emptySeparator
  \experience
  {Nov 2012} {C/C++ Engineer Internship}{R\&D}{State Street (Hangzhou), China}
  {Jul 2012 }    {
  	\begin{itemize}
  		\item  Responsible for the maintenance and development of Princeton Financial Systems. 
  		\item As well as in charge of improving the performance of the system by integrating new technologies.
  	\end{itemize}
  }
  {C/C++ programming, C performance optimization, portfolio}
  	

  \emptySeparator
  \experience
  {Jul 2012} {Member of project team}{Institute of intelligent systems}{ZJUT}
  {May 2011 }    {
				  	\begin{itemize}
				  		\item  Oct 2011-May 2012, write a paper \emph{Traffic routing algorithm based on the spatial complex networks};
				  		\item  May-Sep 2011, work on the project: \emph{Motion Sensing PPT based on Kinect} | \emph{Programmer}.
				  	\end{itemize}
				  }
				  {complex networks, kinect, C\#}
	  				  
  \emptySeparator
  \experience
  { Dec 2011} {\emph{Tiny Software development}}{ZJUT}{C/C++/JAVA}
  {Oct 2011 }    {
				  	\begin{itemize}
				  		\item Oct-Dec 2011, \emph{Online Works Show Platform} | \emph{Leader}. I designed and implemented a lightweight 
						  		relational object JDBC package, which is used for the programming of the server. 		  		
						  		Got the 2\textsuperscript{nd} place of the contest judged by the TaoBao UED.     
						  		\faGithub: \link{https://github.com/huajh/showplatform}{github.com/huajh/showplatform}                                                                           
				  		\item Nov 2011, \emph{Unix File System} | \emph{Independent developer}. The system is implemented by the C/C++. It has basic shell commands, well performed
						  		memory management,  as well as the users management, and it supports parallel operation.
						  		\faGithub: \link{https://github.com/huajh/unix_file_sys}{github.com/huajh/unix\_file\_sys}
				  	\end{itemize}
				  }
				  {JAVA, Unix, software development, Database, Sql Server}
	
		
\end{experiences}
